\documentclass[10pt,a4paper]{article}
\usepackage[latin1]{inputenc}
\usepackage{amsmath}
\usepackage{amsfonts}
\usepackage{amssymb}
\author{Ana Huaman}
\title{Learning Statement \\ Intro to Robotics - CS 7085}

\begin{document}
\maketitle

%%%%%%%%%%%%%%%%%%%%%%%%%%
% Mechanics
\section{Mechanics}
This section exposed us to the basics of Mechanics, starting with the analysys of planar robots and later on extending this to the general spatial case. Initially we learned fundamental concepts such as \textit{Degrees of Freedom, Mobility, Types of manipulators, etc}. Later, we learned how to do \textit{modelling}: to describe a robotic system as a set of free rigid bodies connected by joints and with certain constraints which limit its movement. Based on this, the course went on explaining the \textit{kinematics}, or the study of the movement of the robots (for both planar and parallel topologies)
\medskip

About \textit{kinematics}, first we learned how to relate the joint positions with the workspace positions, by using \textit{Forward Kinematics} and then the inverse case: How to obtain the required joint positions given a desired workspace configuration (\textit{Inverse Kinematics}). Then, analogously, we explored the relationship between velocities, in particular the use of the Jacobian to convert the velocities from configuration space to workspace. This was a particularly important topic, since we learned about \textit{singular configurations} and how they should be considered in our system design (either to avoid them or to plan ahead how the system should respond).
\medskip

Finally, we surveyed some common robotic platforms and the advantages and disadvantages of each of them, depending of our applications. I found this section very useful since the design of a robotic system can be done in a more intelligent way if we know the basics of how these systems work (probably I should take a course to learn more about Dynamics, tough)

%%%%%%%%%%%%%%%%%%%%%%%%%%
% Controls
\section{Controls}
Here more cool stuff! (probably I say that because this is one of my core areas). We had lectures for \textit{Linear} and \textit{Non-Linear Control} so I will explain briefly what I learned from each of them:

\begin{itemize}
\item{\textbf{Linear Systems and Control:} We started by learning how to represent a dynamic system with a \textit{state-based representation}} (Modern Control) and some basics of modeling mechanical systems based on their differential equations. Then, we learned the concepts of \textit{Stability} and how we must make our system to reach a stable state by the use of \textit{controllers}.
\medskip

Naturally, the next step was to learn how to design controllers. We learned how to design \textit{state feedback controllers} for such purpose. However, since our system only give us \textit{output} information, we also learned that we need to \textit{estimate} the state from the output, which means that \textit{Observer Design} must be taken into account. Some more concepts about \textit{Controllability} and \textit{Observability} were introduced to further analyze the properties of the system (and the feasibility of applying the mentioned controls).
\medskip

Conclusion: Control rocks.

\item{\textbf{Non-Linear Control:} }
The first thing we learned is that Linear Controls is an abstraction that only exists in the theoretical space of control guys. Meaning: The whole world is non-linear! Hence, it makes sense to analyze how the real systems (with no linearizations around equilibrim points) work.
\medskip

We revisited the stability concepts for non-linear systems. Since our main goal as control designers is to make our controlled system stable, we learned about \textit{Lyapunov functions} and how to use them to derive a stability form for a dynamic system. Concerning specifically control design, we learned the basics of \textit{feedback linearization}, which can virtually \textit{dissappear} the non-linearities of our system, although to apply it we need to have a system with an accurate model, otherwise this may not work.
\medskip

Non-Linear control is a pretty big area, so I think I should take the class to learn more about it!

\end{itemize}


%%%%%%%%%%%%%%%%%%%%%%%%%%
% Perception
\section{Perception}
To control a system we need some information about the environment, meaning, we need to \textit{perceive} what happens around us. We started this topic with a short survey of the most common sensors in robotics and how they can be used to infer information from the world. Furthermore, since we are talking about physical devices (such as radars, sonars, cameras, etc), we know that the information we get is subject to noise and is imperfect. Hence, we also discussed the use of Bayesian networks and diverse kind of filters to extract useful information from our input data (being point clouds, imagery, etc).
\medskip

Depending of the application, we might use perception for different goals. For instance, if our application includes a robot that  knows the map of its location, but does not know exactly where in the map it is, we can make the robot to explore the cited location and use the collected \textit{perception data}(such as sonar info) to infere, through time, what its current position is. The problem becomes more interesting when the robot actually \textit{does not know what its environment is beforehand} (let alone where in the environment the robot is located!).This constitutes the \textit{SLAM problem}, which we analyzed at some extent at class.

Finally, we also explored Perception applied with images, such as \textit{Sterero Vision}, which allow us (given images from two points of view) to infere important information from our environment, such as 3D depth and how far or near is an object, which is important if we plan our system to do something useful (let's say, motion planning). Other applications we discussed were perception applied to robot soccer, in which the perception input is vital to help the agent \textit{decide} what action to do next, based on the current state of the world.
\medskip

Conclusion: I hope to take a Computer Vision class soon! (I have taken the undergrad class this semester tough)

%%%%%%%%%%%%%%%%%%%%%%%%%%
% Artificial Intelligence
\section{Artificial Intelligence / Autonomy}
More interesting stuff here. So far, we had learned how to perceive our environment, how to get information out of it and how to control a mechanical system to do what we want. What we need now is how to make our system \textit{autonomous}: A system reliable enough such that it could take the best decisions towards reaching its target state.

We had an interesting overview of how important autonomous robots are, specially for applications in which the environment is hard to access and a human supervisor is not a viable option (i.e. we saw an example of a robot exploration in the Antartic, in which the weather made it nearly impossible for a robot to have a human on-board). Autonomy in fact, implies the integration of all the other tools (such as Perception and Control) to make our system able to reach its goals.

\section{Human-Robot Interaction}
This was the last area we investigated. Human-Robot Interaction is a very active research area that involves the study of how humans and robotic systems interact, how to make the cited interaction more beneficial and more fluent. There are multiple applications which can be studied:

\begin{itemize}
\item{Use of robot as teachers for toddlers}
\item{Use of robots for assistance of elderly}
\item{Use of service robots in offices, museums, etc}
\item{Research area in general: Robots for remote game playing (playing tennis, etc)}
\item{Use of \textit{human teachers} to teach skills to the robots such that they can imitate human behaviors}

All the activities do not necessarily involve the robot in a Laboratory environment, meaning that the robots are intended to be used for people in a daily basis in home-like environments. Hence, one important aspect in HRI is to make sure that the robots present a friendly interface such that it motivates people to use them in the most effective way.
\end{itemize}

\section{Integration and Summary}
My core research areas are: \textit{Controls, Artificial Intelligence and Perception}, since my specific research topic is \textit{Path Planning for Highly Dimensional Systems}. I am developing planning algorithms for robotic arms, for which naturally I am using the knowledge obtained by:

\begin{itemize}
\item{\textit{Controls:} To control the physical robotic arm.}
\item{\textit{Artificial Intelligence:} Planning Algorithms.}
\item{\textit{Perception:} To perceive the environment in which the arm moves and update the control and planning strategy}
\end{itemize}

To be honest, I am still not sure of how to use the other areas yet, since I have not taken classes yet for these, but I am sure they will be useful. Possibly:

\begin{itemize}
\item{\textit{Mechanics:} For the kinematics and dynamics analysis of the robotic arm. Actually, I have already made some kinematics calculations for the control of the End Effector of the arm. Probably I will need to learn more dynamics to apply force control to grasp objects.}
\item{\textit{Human-Robot Interaction:} Eventually, the planning of paths for robotics arms should be used in home service robots, hence, in some way some HRI will be involved. For instance, to develop the paths, a factor to consider is that these paths should look natural(i.e. the robot arm should not bend in weird angles, that are not normal for humans when performing the same activity).}
\end{itemize}
\end{document}